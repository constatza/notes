\documentclass[a4paper, twocolumn]{article}

\usepackage{polyglossia,hyphenat}
\setmainlanguage{greek}

\usepackage{fontspec}
\setmainfont{Linux Libertine O}
\setsansfont{Linux Biolinum O}
\usepackage{amsmath}
\usepackage{amssymb}
\usepackage{physics}


\title{Μητρωική Στατική\\
\Large Σημειώσεις}
\date{}


\newcommand{\num}[1]{ N_{\mathit{#1}} }
\newcommand{\R}[1]{ \mathbb{R}^{#1} }
\newcommand{\gs}{\mathcal{G}}
\newcommand{\ls}{\mathcal{L}_i}
\newcommand{\vect}[1]{ \{ #1\} }
\newcommand{\mat}[1]{\left[ #1 \right]}
\newcommand{\gtol}{\mat{\Lambda^i}}

\newcommand{\lstiff}[1]{\mat{k^{\mathit{#1}}}}
\newcommand{\gstiff}[1]{\mat{\bar{k}^{\mathit{#1}}}}
\newcommand{\Gstiff}[1]{\mat{\bar{K}^{\mathit{#1}}}}
\newcommand{\gforce}[1]{\vect{\bar{P}^{\mathit{#1}}}}
\newcommand{\gdisp}[1]{\vect{\bar{D}^{\mathit{#1}}}}

\newcommand{\subk}[1]{ \mat{\bar{K}_{\mathit{#1}}} }
\newcommand{\subp}[1]{\vect{\bar{P}_{\mathit{#1}}}}
\newcommand{\subd}[1]{\vect{\bar{D}_{\mathit{#1}}}}

\begin{document}
\maketitle

\section{Βασική μέθοδος}

\subsection{Συστήματα αναφοράς}
Πριν ξεκινήσουμε οποιαδήποτε διαδικασία επίλυσης, μελετούμε καλά τον 
φορέα και επιλέγουμε \emph{τύπο} στοιχείων/μελών ανάλογα με το 
πρόβλημα που μας δίδεται (π.χ. στοιχείο δοκού 2Δ, δικτυώματος 3Δ 
κλπ.).
\begin{enumerate}
	\item Oρίζουμε το \emph{Καθολικό Σύστημα Αναφοράς} $\gs$.
	\item Αριθμούμε τους \emph{κόμβους}(=\emph{nodes}) της 
	κατασκευής, $n=1, 
	\dots,\num{nodes}$.
	\item Αρίθμούμε τα \emph{στοιχεία}(=\emph{elements})\footnote{
	οι όροι \emph{στοιχείo}, \emph{μέλος} και οι αντίστοιχοι αγγλικοί 
	όροι \emph{element} και \emph{member} είναι απολύτως ισοδύναμοι 
	όταν αφορούν την υποδιαίρεση ενός φορέα σε τμήματα.} του φορέα, 
	$i=1, \dots, \num{elements}$.
	\item Αριθμούμε τους καθολικούς \emph{βαθμούς ελευθερίας}(=β.ε. ή 
	dofs), $d=1, \dots, \num{dofs}$,
	των κόμβων της κατασκευής.

	\item Ορίζουμε το \emph{τοπικό σύστημα αναφοράς} $\ls$ 
	για κάθε στοιχείο.
\end{enumerate}

\subsection{Μεγέθη στοιχείου}
Οι διαστάσεις των μεγεθών του στοιχείου 
εξαρτώνται από τον τύπο του 
μέλους (δοκός2Δ, δικτύωμα3Δ κλπ.) και συγκεκριμένα 
από τον πλήθος των 
β.ε. $\num{dofs}^i$, δηλαδή
\begin{itemize}
	\item κάθε διάνυσμα στοιχείου $\vect{w^i}$ θα έχει διάσταση 
	$\num{dofs}^i \times 1$.
	\item κάθε τετραγωνικός πίνακας στοιχείου $\mat{W^i}$ θα έχει 
	διαστάσεις $\num{dofs}^i  \times \num{dofs}^i$
\end{itemize}

Για κάθε στοιχείο $i$, λοιπόν, του 
φορέα, χρησιμοποιώντας τις φυσικές και γεωμετρικές σταθερές που το 
χαρακτηρίζουν (π.χ.$E_i, L_i, \theta_i$ κλπ.) μορφώνουμε
\begin{enumerate}
	\item το μητρώο μετασχηματισμού από το ΚΣ στο ΤΣ
	$ \gtol: \gs \rightarrow \ls ,$
	\item το μητρώο στιβαρότητας του στοιχείου στο τοπικό 
	$\lstiff{i}$ και
 	στo καθολικό σύστημα
		\begin{equation}
			\gstiff i = \gtol^T \lstiff{i} \gtol \in 
			\mathbb{R}^{\num{dofs}^i \times \num{dofs}^i}
		\end{equation}

	\item το διάνυσμα των αντιδράσεων παγιώσεως του μέλους στο τοπικό 
	σύστημα $\vect{A_r^i} $ και στο καθολικό 
	\begin{equation}
		\vect{\bar A_r^i} = \gtol^T \vect{A_r^i},
	\end{equation}
	αν στο μέλος ασκούνται ενδιάμεσες δράσεις (π.χ. 
	συγκεντρωμένα/κατανεμημένα φορτία, θερμοκρασιακά κλπ.)

\end{enumerate}

Οι διαστάσεις των μεγεθών του στοιχείου 
εξαρτώνται από τον τύπο του 
μέλους (δοκός2Δ, δικτύωμα3Δ κλπ.) και συγκεκριμένα 
από τον αριθμό των β.ε. τους μέλους $\num{dofs}^i$. Έτσι, 
\begin{itemize}
	\item κάθε διάνυσμα στοιχείου $\vect{w^i}$ θα έχει διάσταση 
	$\num{dofs}^i \times 1$.
	\item κάθε τετραγωνικός πίνακας στοιχείου $\mat{W^i}$ θα έχει 
	διαστάσεις $\num{dofs}^i  \times \num{dofs}^i$
\end{itemize} 

% GLOBAL MATRICES 
\subsection{Μεγέθη φορέα}
Αφού ολοκληρώσαμε τα προαναφερθέντα σε επίπεδο μελών, ξεκινάμε την 
μόρφωση του συνολικού φορέα.

\begin{enumerate}
	\item Kατασκευάζουμε το καθολικό μητρώο στιβαρότητας $\Gstiff{}$
	 αθροίζοντες σε κάθε καθολικό β.ε. της κατασκευής την στιβαρότητα 
	 των συμβαλλόμενων β.ε. κάθε μέλους
	\begin{equation}
		\subk{ij} = \sum_{i=1}^{\num{elements}} \mat{v^i} \gstiff{i} 
		\mat{v^i}^T
	\end{equation}
	O πίνακας $\mat{v^i}$ τού μέλους $i$ έχει διαστάσεις $\num{dofs} 
	\times \num{dofs}^i$ και κάθε στοιχείο του $v^i_{\mathit{lm}}$ 
	του μέλους $i$ ισούται με την 
	μονάδα, αν o $l$ β.ε. της κατασκευής αντιστοιχεί στον $m$ β.ε.του 
	μέλους. Σε κάθε άλλη περίπτωση είναι μηδενικό.
	
	\item Κατασκευάζουμε το διάνυσμα των δράσεων παγίωσης, 
	αθροίζοντας τα αντίστοιχα διανύσματα μελών στους καθολικούς β.ε. 
	που αντιστοιχούν. Οι συνιστώσες του διανύσματος δράσεων παγίωσης 
	είναι
	\begin{equation}
		\vect{\bar S} = \sum_{i=1}^{\num{elements}} \mat{v^i} 
		\vect{\bar{A}_r^i}
	\end{equation}
	με τρόπο που ομοιάζει με του μητρώου στιβαρότητας.
	
	\item Μορφώνουμε τα διανύσματα επικόμβιων δράσεων $\vect{\bar 
	P^{\mathit{nodal}}}$ και μετακινήσεων $\vect{\bar D}$. Το 
	διάνυσμα 
	των
	τελικών δράσεων στους κόμβους προκύπτει αφού αφαιρέσουμε και τις 
	δράσεις παγιώσεως
	\begin{equation}
		\vect{\bar P} = \vect{\bar P^{\mathit{nodal}}} - \vect{\bar S}
	\end{equation}
	\item Αν χρειαστεί, σχηματίζουμε το μητρώο περιστροφής λόγω 
	κεκλιμένων στηρίξεων $\mat{R}$. Τα τροποποιημένα μεγέθη γίνονται
	\begin{align}
		&\gforce{m} = \mat{R}^T \gforce{} \\
		&\gdisp{m} = \mat{R}^T \gdisp{} \\
		&\Gstiff{m} = \mat{R}^T \Gstiff{} \mat{R}
	\end{align}
	Αν ο φορέας δεν έχει κεκλιμένη στήριξη, τότε πίνακας περιστροφής 
	ισούται προφανώς με τον μοναδιαίο πίνακα $\mat{R}=I$.
	
	\item Σχηματίζουμε το μητρώο αναδιάταξης $\mat{V}$ για να 
	χωρίσουμε τους δεσμευμένους β.ε. από τους ελεύθερους. Τα 
	διατεταγμένα μεγέθη γίνονται
	\begin{align}
		&\gforce{mm} = \mat{V}^T \gforce{m} \\
		&\gdisp{mm} = \mat{V}^T \gdisp{m} \\
		&\Gstiff{mm} = \mat{V}^T \Gstiff{m} \mat{V} 
	\end{align}
\end{enumerate}

\subsection{Επίλυση Φορέα}
Η τελική εξίσωση ισορροπίας μετά τις τροποποιήσεις γράφεται
	\begin{equation}\label{eqn:mm}
		\gforce{mm} = \Gstiff{mm} \gdisp{mm}
	\end{equation}
Αναλύοντας την \ref{eqn:mm} στα αντίστοιχα υπομητρώα με βάση τους 
\emph{ελεύθερους} (free) και \emph{δεσμευμένους} (supported) β.ε. 
λαμβάνουμε
	\begin{equation}
		\mqty[ \subp{f} \\ \subp{s} ] = 
		\mqty[ \subk{ff} & \subk{fs} \\
		\subk{sf}& \subk{ss} ] 
		\mqty[ \subd{f} \\ \subd{s} ]
	\end{equation}
και εκτελώντας τον πολλαπλασιαμό των υποπινάκων καταλήγουμε στις
	\begin{align}
	 	&\subp{f} = \subk{ff} \subd{f} + \subk{fs} \subd{s} 
	 	\label{eqn:first} \\
		&\subp{s} = \subk{sf} \subd{f} + \subk{ss} \subd{s}. 
		\label{eqn:second}
	\end{align}
Λύνοντας την \ref{eqn:first} ως προς τις ελεύθερες 
μετακινήσεις παίρνουμε την έκφραση
	\begin{equation}
		\subd{f} 
		= \subk{ff}^{-1} ( \subp{f} - \subk{fs} \subd{s} ) 
		\label{eqn:dfree} 
		%\\
		%&\subp{s} = \subk{sf} \subd{f} + \subk{ss} \subd{s}
		%\label{eqn:psupp}
	\end{equation}
την οποία έπειτα αντικαθιστούμε στην 
\ref{eqn:second} για να βρούμε και τις άγνωστες αντιδράσεις στους 
δεσμευμένους β.ε.~.

\subsection{Εντατικά μεγέθη}
Από την στιγμή που έχουμε βρει τις άγνωστες μετακινήσεις και 
αντιδράσεις μπορούμε εύκολα να υπολογίσουμε και την ένταση που 
ασκείται στους βαθμούς ελευθερίας κάθε μέλους.

\begin{enumerate}
	\item Αρχικά μετατρέπουμε τις τροποποιημένες μετακινήσεις της 
	κατασκευής σε καθολικές
	\begin{equation}
		\gdisp{} =\mat{R} \mat{V} \gdisp{mm}.
	\end{equation}
		
	\item Έπειτα, βρίσκουμε ποιες καθολικές μετακινήσεις του φορέα, 
	αντιστοιχούν σε κάθε μέλος $i$
	\begin{equation}
		\gdisp{i} = \mat{v^i}^T \gdisp{}.
	\end{equation}
	Οι διαστάσεις του $\gdisp{i}$ είναι $\num{dofs}^i \times 1$.
	
	\item Η τελική ένταση στους β.ε. του μέλους προκύπτει από την 
	επαλληλία των τοπικών δράσεων του μέλους του παγιωμένου και του 
	ισοδύναμου φορέα.
	\begin{equation}
		\vect{A^i} = \vect{A^i_r} + \vect{A^i_e}
	\end{equation}
	όπου οι δράσεις του ισοδύναμου φορέα δίνονται από την
	\begin{equation}
		\vect{A^i_e} = \lstiff{i} \mat{D^i}
		= \lstiff{i} \mat{\Lambda^i} \gdisp{i}.
	\end{equation}
\end{enumerate}

\end{document}